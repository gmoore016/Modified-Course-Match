\documentclass{beamer}

\usetheme{metropolis}

\title{Course Allocation and Modified CEEI}
\author{Gideon Moore}
\date{Fall 2019}

\usepackage{microtype}

\usepackage[backend=biber, style=authoryear, maxbibnames=99]{biblatex}

\addbibresource{coursematch.bib}

\setbeamertemplate{section in toc}[sections numbered]

\begin{document}
\begin{frame}
\maketitle
\end{frame}
\begin{frame}{Outline}
\tableofcontents
\end{frame}
\section{The Course Allocation Problem}
\begin{frame}{Problem Outline}
\begin{itemize}
    \item University needs to decide how to allocate students to courses
    
    \item Students enroll in multiple courses, and courses contain multiple students
    
    \item Students have preferences over \emph{bundles} of courses
    
    \item Courses are indivisible
\end{itemize}
\end{frame}
\begin{frame}{Desirable Properties}
\begin{itemize}
    \item Incentive Compatible
    \pause
    \item Uncongested
    \pause
    \item Efficient
    \pause
    \item Fair
    \pause
    \item \textbf{Two-Sided}
\end{itemize}
\end{frame}
\begin{frame}{Existing Solutions}
\begin{itemize}
    \item Free-For-All
    \pause 
    \item Lottery 
    \pause 
    \item Serial Dictatorship \parencite{klaus2002}
    \pause 
    \item Draft \parencite{budish2012}
    \pause 
    \item Point Auction \parencite{sonmez2010}
\end{itemize}
\end{frame}

\section{Competitive Equilibrium from Equal Incomes}
\begin{frame}{\textcite{budish2011}}
\begin{itemize}
    \item Students endowed with ``approximately'' equal incomes
    \item Course bundles are ranked \parencite{budish2016b}
    \item \textcite{othman2010} algorithm finds prices
\end{itemize}
\end{frame}
\begin{frame}{A-CEEI Properties}
\begin{itemize}
    \item Pareto efficient
    \pause
    \item Uncongested
    \pause
    \item \emph{Approximately} incentive compatible
    \pause
    \item \emph{Approximately} fair
    \pause
    \item Single sided
\end{itemize}
\end{frame}
\begin{frame}{Two-Sidedness and Externalities}
Suppose each student-course matching imposed some externality $s_{ij}$ on students and faculty.

Let $\vec{x}_i$ be a vector of booleans indicating student $i$'s enrollment status in each of $M$ courses.
\pause

Budish's A-CEEI maximizes $$\sum_{i \in \mathcal{S}} \mathcal{U}_i(\vec{x}_i)$$
\pause

However, a social planner would maximize $$\sum_{i \in \mathcal{S}}\left( \mathcal{U}_i(\vec{x}_i) + \vec{x}_i^T \vec{s}_i \right)$$
\end{frame}
\begin{frame}{Proposed Mechanism}
Administrators can correct externalities by imposing taxes on the course market.

Each student-course pair is assigned some tax rate $k_{ij}$ such that student $i$ faces price $k_{ij} \cdot p_j$ when attempting to purchase course $j$.
\end{frame}
\begin{frame}{Properties of Modified CEEI}
    \begin{itemize}
        \item Incentive Compatible
        \pause 
        
        \item Uncongested
        \pause 
        \item Envy-bounded within a bracket
        \pause 
        
        \item Pareto efficient within a bracket
        \pause 
        
        \item Two-sided
    \end{itemize}
\end{frame}
\begin{frame}{Optimal Taxation}
Economics 1 tells us to tax each student's enrollment at a rate equal to $s_{ij}$
\pause 

This doesn't work here due to \emph{lack of an outside good}
\pause

Severing of price from marginal utility makes optimal taxation a case-by-case question
\end{frame}
\begin{frame}{Types of Tax Schema}
Let each student face a vector of tax rates $\vec{k}_i$

\begin{itemize}
    \item All students face identical constant tax rate
    \pause 
    
    \item Tax vectors all same direction, varying magnitudes \parencite{miralles2014}
    \pause 
    
    \begin{itemize}
        \item Serial dictatorship: $k_i \in \left((1 + \mbox{largest permissible bundle size})^{-i}\right)_{i = 1}^N$
    \end{itemize}
    \pause 
    
    \item Tax vectors vary in direction
\end{itemize}
\end{frame}


\section{Market Clearing Error}
\begin{frame}{\textcite{othman2010}}
Given a course preference and endowment for each student, can generate market clearing prices \pause (kinda)
\pause 

Each student has at most $M$ budget constraints 
\pause 

Income dispersion means that in base version, at most $M$ constraints intersect at a single point
\pause 

Therefore, there are at most $2^M$ demands adjacent to a single point
\pause 

Demand across a budget constraint shifts by at most $\sqrt{\sigma} = \sqrt{\min(M, 2 \cdot \mbox{max schedule size})}$
\end{frame}
\begin{frame}{\textcite{othman2010} Illustration}
    \centering
    \includegraphics[width=.6\textwidth]{"Price Discontinuity"}
\end{frame}

\begin{frame}{Error and the Convex Hull}
We know there is some ideal demand within the convex hull of the feasible demands

Thus, the worst-case market clearing error is the maximum distance from a feasible demand to a point in the convex hull

Given there are $2^M$ possible points, the worst case is a hypercube with ideal demand equidistant from all vertices. Half the diagonal is $\frac{\sqrt{M\sigma}}{2}$, which is our market clearing error.
\end{frame}

\begin{frame}{Generalizing with Tax Rates}
Within a given tax rate, we can apply the basic proof to show at most $M$ constraints intersect
\pause 

However, we know nothing about constraints under \emph{other} tax rates
\pause 

Thus, the maximum size of an intersection scales with the maximum number of tax brackets which apply to a single course $h$, with a maximum of $hM$ intersecting planes. In the worst case, each plane is duplicated in each other tax bracket, in which case there exist $2^M$ demands around $p$, but each discontinuity is of magnitude $h\sqrt{\sigma}$.

This again yields a hypercube, now with side length $h\sqrt{\sigma}$, and therefore a diagonal measurement of $h\frac{\sqrt{M\sigma}}{2}$
\end{frame}
\begin{frame}{Empiric Results}
When implemented at Wharton, true market clearing demand was significantly less than the bound.

On average across the four semesters studied, fewer than six seats total were misallocated.
\end{frame}
\begin{frame}{Next Steps}
	\begin{itemize}
		\item Continuous setting--move to continuous mass of students of each type. Has some advantages:
		\begin{itemize}
			\item No need to fuss with error bounds
			\item Results are more likely to be analytically tractable
			\item Students are \emph{definitionally} price takers
			\item Equilibrium guaranteed to exist
		\end{itemize}
		\pause
		\item Tax setting--do these "tax rates" have \emph{any} sort of external meaning?
		\pause
		\item What if we want a heterogeneous class?
		\pause
		\item Can this handle quotas as well as taxes?
		\pause
		\item Is there a clean way to handle the ``eccentric student'' problem?
		\pause
		\item For an empirical approach, find one \emph{specific, meaningful} externality to address
	\end{itemize}
\end{frame}
\begin{frame}[allowframebreaks]{Bibliography}
\printbibliography
\end{frame}
\end{document}
